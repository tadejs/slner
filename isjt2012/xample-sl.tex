\documentclass[10pt, a4paper]{article}
\usepackage{lrec2000-sl}           % predpisana oblika organizatorja
\usepackage[slovene]{babel}     % slovensko deljenje besed
\usepackage[latin2]{inputenc}   % vnos znakov v zapisu ISO 8859-2

\title{Primer �lanka za konferenco IS-LTC}

\name{Primo� Peterlin$^\ast$, Toma� Erjavec$^\dagger$, Ale� Ko�ir$^\ddagger$}

\address{ $^\ast$In�titut za biofiziko, Medicinska fakulteta,
               Univerza v Ljubljani \\
               Lipi�eva 2, 1000 Ljubljana\\ 
               primoz.peterlin@biofiz.mf.uni-lj.si \\ 
               \\
               $^\dagger$Odsek za tehnologije znanja \\
               Institut ``Jo�ef Stefan''\\
               Jamova 39, 1000 Ljubljana\\
               tomaz.erjavec@ijs.si \\ 
               \\
               $^\ddagger$Hermes SoftLab \\ 
               Litijska 51, 1000 Ljubljana\\
               ales.kosir@hermes.si
               \\
        }

\abstract{Sem pride povzetek. Sem pride povzetek. Sem pride povzetek.
  Sem pride povzetek. Sem pride povzetek.  Sem pride povzetek. Sem
  pride povzetek. Sem pride povzetek. Sem pride povzetek. Sem pride
  povzetek. Sem pride povzetek. Sem pride povzetek. Sem pride
  povzetek.  Sem pride povzetek. Sem pride povzetek. Sem pride
  povzetek.}

\begin{document}

\maketitleabstract

\section{Uvod}

Prva vrstica vsakega odstavka je zamaknjena za 0.5 cm.  Prva vrstica
vsakega odstavka je zamaknjena za 0.5 cm.  Prva vrstica vsakega
odstavka je zamaknjena za 0.5 cm.

Prva vrstica vsakega odstavka je zamaknjena za 0.5 cm. Prva vrstica
vsakega odstavka je zamaknjena za 0.5 cm.


\section{Namen �lanka}

Opis namena �lanka. Opis namena �lanka. Opis namena �lanka. Opis
namena �lanka. Opis namena �lanka. Opis namena �lanka. Opis namena
�lanka.


\subsection{Primer podpoglavja}

Primer podpoglavja. Primer podpoglavja. Primer podpoglavja. Primer
podpoglavja. Primer podpoglavja. Primer podpoglavja.

Primer podpoglavja. Primer podpoglavja. 


\subsubsection{Primer podpodpoglavja}

�e en primer, tokrat podpodpoglavja. �e en primer, tokrat
podpodpoglavja. �e en primer, tokrat podpodpoglavja. �e en primer,
tokrat podpodpoglavja. �e en primer, tokrat podpodpoglavja. 


\subsubsection{Primer podpodpoglavja z dolgim naslovom, ki se
  razteza preko dveh vrstic}

�e en primer podpodpoglavja. �e en primer podpodpoglavja. �e en primer
podpodpoglavja. �e en primer podpodpoglavja. �e en primer
podpodpoglavja. 


\section{Dodatna navodila}

\subsection{Opombe}

To je primer opombe.\footnote{To je besedilo opombe.}

\subsection{Slike}

Primer slike v okviru.

\begin{figure}[h]
\begin{center}
\fbox{\parbox{6cm}{
To je slika z naslovom. To je slika z naslovom. To je slika z naslovom. }}
\caption{Naslov slike.}
\end{center}
\end{figure}

\subsection{Tabele}

Razlikujemo dva tipa tabel, take znotraj stolpca in take, ki so
prevelike za v stolpec.

\subsection{Tabela znotraj stolpca}

Tu je zgled tabele znotraj stolpca.

\begin{table}[h]
 \begin{center}
\caption{Naslov tabele.}
\label{tab:mala}
\medskip
\begin{tabular}{|l|l|}

      \hline
      Nivo&Orodja\\
      \hline\hline
      Oblikoslovje & analizator INTEX\\
      Skladnja & analizator ALE\\
      Semantika & analizator ALE\\
      & skupaj z uporabo modela DRT\\
      \hline

\end{tabular}
 \end{center}
\end{table}

\subsection{Velike tabele}

Tabela \ref{tab:velika} je zgled velike tabele.
Ta odplava na naslednjo stran.

\begin{table*}[ht]
 \begin{center}
\caption{Naslov velike tabele.}
\label{tab:velika}
\medskip
\begin{tabular}{|l|l|}

      \hline
      Nivo&Orodja\\
      \hline\hline
      Oblikoslovje & analizator INTEX\\
      Skladnja & analizator ALE\\
      Semantika & analizator ALE
      skupaj z uporabo modela DRT\\
      \hline

\end{tabular}
 \end{center}
\end{table*}


\section{Oblikovanje literature}

Vse bibliografske reference v besedilu so v oklepajih, ki vsebujejo
priimek avtorja in, za vejico, letnico izida \cite{chomsky-73}. 
�e stavek �e vsebuje ime avtorja, potem naj oklepaji vsebujejo samo
letnico: \newcite{aslin-49}.
Ko navajamo ve� del, so referece lo�ene s podpi�jem:
\cite{chomsky-73,feigl-58}.
�e ima referenca ve� kot tri avtorje, potem vidimo samo ime prvega,
�emur sledi et al. \cite{fletcher-hopkins}.

Bibliografski viri so na�teti po abecednem vrstnem redu na koncu
�lanka. Naslov poglavja je ``Literatura'' in je vrhnje poglavje.  Prvi
vrstica vsake bibliografske reference je poravnana levo v stolpcu,
ostanek reference pa je zamaknjen za 0.35 cm.  Naslednji primeri
ponazarjajo format za zbornike konferenc \cite{chave-64}, knjige
\cite{butcher-81}, �lanke v revijah \cite{howells-51}, doktorske
disertacije \cite{croft-78} in poglavja v knjigah \cite{feigl-58}.

%\nocite{*}

\bibliographystyle{lrec2000-sl}
\bibliography{xample-en} 
%%%% Ta ne dela povsem:
%\bibliography{xample-sl} 

\end{document}
